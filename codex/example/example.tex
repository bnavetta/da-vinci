\documentclass{article}

\usepackage{fontspec}
\defaultfontfeatures{Ligatures=TeX}

\usepackage{pythontex}

\begin{pycode}
import da_vinci

da_vinci.init(pytex)
da_vinci.init_tables()
da_vinci.init_figures()

print(da_vinci.preamble())

import matplotlib.pyplot as plt
import numpy as np
import pandas as pd

df = pd.DataFrame({
    'foo': np.random.rand(10),
    'bar': np.random.rand(10)
})
data = da_vinci.Data(df, names = {'foo': 'Foo', 'bar': 'Bar'})

\end{pycode}
\printpythontex

\begin{document}

\section{Tables}
Da Vinci can generate \LaTeX{} tables from Data objects. A Data object combines a Pandas DataFrame with display information such as human-friendly
column names and formatters.

\py{da_vinci.latex_table(data)}
% \pyc{print(da_vinci.latex_table(data))}

\section{Figures}

\begin{pycode}
plt.scatter(data["foo"], data["bar"], c="b")
print(da_vinci.latex_figure("foo_bar_plot"))
\end{pycode}

Da Vinci also helps displaying PyPlot figures. The figures are saved into the generated file directory and tracked using PythonTex's created-file support.



% Also add: snippets for useful things like a header suitable for a school assignment, hiding section and page numbers, etc.

\end{document}
